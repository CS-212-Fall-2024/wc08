\documentclass[a4paper]{exam}

\usepackage{amsmath,amssymb, amsthm}
\usepackage{geometry}
\usepackage{graphicx}
\usepackage{hyperref}
\usepackage{titling}


\newcommand{\classX}[1]{\ensuremath{\text{\textsf{\textbf{#1}}}}} 
\newcommand{\classL}{\classX{L}}

% \theoremstyle{definition}
\newtheorem{definition}{Definition}


% Header and footer.
\pagestyle{headandfoot}
\runningheadrule
\runningfootrule
\runningheader{CS 212, Fall 2024}{WC 08: Undecidability}{\theauthor}
\runningfooter{}{Page \thepage\ of \numpages}{}
\firstpageheader{}{}{}

% \printanswers %Uncomment this line

\title{Weekly Challenge 08: Undecidability}
\author{Blingblong} % <=== replace with your student ID, e.g. xy012345
\date{CS 212 Nature of Computation\\Habib University\\Fall 2024}

\qformat{{\large\bf \thequestion. \thequestiontitle}\hfill}
\boxedpoints

\begin{document}
\maketitle

\begin{questions}
  
\titledquestion{The Biryani Theorem}
    Raahim and Taqi got way too excited with their Nature of Computation course and the limits of computation. They keep showing each other cool undecidable problems to mog each other.
    In all this, their friend Roshan Zehra notices that there is a pattern in all the undecidable problems that Raahim and Taqi keep bringing. Whether it is from undecidability of \href{https://arxiv.org/abs/1904.09828}{Magic: The Gathering} to $E_{TM}$ there seem to be a pattern, that is that most of these problems are concerned with some questions regarding languages of Turing machines which are concerned with some general feature or quality regarding the language. Asking questions like ``is the language of this Turing machine Empty'' or ``is the language of this Turing machine Regular''. When discussing this with their friend Musab they conjecture that all such problems must be undecidable.

    Deciding a question $Q$ regarding the language of a Turing machine means when given a Turing machine $M$ we ask $Q$ regarding the language of $M$, if the answer is ``yes'' then we accept $M$ if the answer is ``no'' we reject. For example, if $Q:$``is the language of this Turing machine Regular'', now if we ask this question regarding a Turing machine whose language is $0^n1^n$ then we reject that Turing machine but if we ask this regarding a Turing machine whose language is $\{0,1\}^*$ we accept that Turing machine.
    Now a question $Q$ regarding the language of a Turing machine is said to be nontrivial if there exists at least one Turing machine whose language gives the answer ``yes'' on $Q$ and there is at least one Turing machine whose language gives the answer ``no'' on $Q$. Show that any such nontrivial question $Q$ regarding languages of Turing machines is undecidable. 
   
    To help them their lovely RA has given a general idea for the proof. ``For any such non trivial question regarding languages of Turing machines suppose the question is decidable then with the right construction you will have that if this question is decidable then a well known undecidable problem will alo be decidable therefore this question should not be decidable''. 
  \begin{solution}
    % Enter your solution here.
  \end{solution}
\end{questions}
\end{document}

%%% Local Variables:
%%% mode: latex
%%% TeX-master: t
%%% End:
